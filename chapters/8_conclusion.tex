\chapter{Conclusions}
\label{chap:conclusion} 

\glsresetall

Out of frustration with the difficulty of capturing and annotating image data, the goal of this thesis became how to effectively annotate images to use machine learning models on new domains in visual recognition. After an exploratory study into the annotation of segmentation datasets (in Chapter~\ref{chap:bootstrap}), the most promising direction was a \emph{human-in-the-loop} annotation method I have termed \gls{VBA}, where a machine learning model is trained interactively based on verifying and correcting predictions. 

\section{Verification based annotation}

The major work in this thesis has been in evaluating and characterising methods centred around \gls{VBA}, intended for human-efficient annotation as well as rapid prototyping. I demonstrate the methods through a web-based \gls{VBA} system for object detection (Chapter~\ref{chap:design}) with a focus on experimentation on new domains. A key point of difference to similar works, either prior or concurrently to this work, is the focus on online training, where existing works use either (a) strong models trained on large data sets or (b) staged systems with alternating periods of annotation and training. 

This work has shown \gls{VBA} to be a practical method for machine-assisted annotation and combines well with the ability to explore and manage image data to be a useful tool for experimentation in new domains. It is especially effective in image sets with uniformity of object instances, reducing required annotation outright by $75$--$93\%$, and a further $10\%$ in several cases, using novel methods for utilising weakly confident detections in a wide variety of real-world annotation efforts. Together with effective image selection, \gls{VBA} offers potential to be applicable to most annotation problems.

I apply \gls{VBA} to verified counting of Ad\'elie penguins, and Weddell seals, where it has the promise of revolutionising the field. Counting takes a fraction of the effort and improves consistency compared to widely used methods such as crowdsourcing. \gls{VBA} offers immediate visual feedback and improved engagement, where the chore of annotating many images becomes the exciting task of teaching a machine to recognise the objects.

\section{Application of \texorpdfstring{\gls{CNN}s}{} to Verification Based Annotation}

A theme during this work, with a direct impact on assisted image annotation, has been how the alignment and resolution of objects in images affects visual recognition performance. An initial study (Chapter~\ref{chap:focus}), looked at how image cropping affected classification performance using a \gls{CNN} for classification. For an instance recognition dataset, a tight cropping and increased resolution were both strongly beneficial to classification using a \gls{CNN}. 

In Chapter~\ref{chap:object_detection}, I continue this work in the domain of \gls{CNN}-based object detection. I propose methods for high-resolution object detection, which improve accuracy and the speed of learning, and study how noise and systematic bias degrade performance. In the presence of small amounts of either, the object detector is quite robust. With reduced data, for example, at the start of an annotation process, the sensitivity to noise and systematic bias increases with the additional problem of overfitting. In Chapter~\ref{chap:annotation}, human thresholds for acceptable noise are established at around $80$ to $85$ \gls{IOU}. To reach that level, moderate to large amounts of noise badly degrade performance. Therefore, a focus should be on accurate annotation, especially at the beginning of the process.

\section {Future work}

Future work will focus on several areas: 

\begin{itemize}
    \item More sophisticated forms of visual recognition: instance segmentation with polygons, curve detection and tree structures, and pose estimation.
    \item Improving the software for broader release, and easy deployment on cloud services.
    \item Improved use of data, using k-fold cross validation ensembles for uncertainty estimation and cross-validating annotated images.
    \item Application to larger scale annotation tasks, and multi-user annotation.
\end{itemize}


 Chapters~\ref{chap:object_detection}~and~\ref{chap:annotation} will form the basis for journal publications. The work on verified counting will be expanded and also published as a larger scale comparison between machine learning and crowdsourcing approaches, as part of a larger collaboration.







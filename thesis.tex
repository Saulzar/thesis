\documentclass[twoside]{uocthesis}

\Title{The Use of the Meta-Syntactic Variable `Foo'}
\Author{Michael JasonSmith}
\Year{2000}

\Supervisor{Dr Andy Cockburn}
\Department{Department of Computer Science}

\begin{document}

\prelimpages

\titlepage

\dedication{To Spiny Norman.}

\abstract{%
The uses of the word `Foo' is discussed with historical context and
modern ramifications stemming from its use as a meta-syntactic
variable.}

\tableofcontents

\acknowledgments{%
I will like to thank all the hedgehogs, without there help this
thesis would not be a reality.}

\textpages

\chapter{Foo in History}
\label{sec:hist}
The first use of `foo' was during World War II, when unknown aircraft
were referred to as `Foo Fighters'.
Another related term was \textsc{Fubar}, for F***ed Up Beyond All
Recognition.
As Eric S. Raymond says in the Jargon File\cite{winder2009picking}:
\begin{quote}
  The etymology of hackish `foo' is obscure. When used in connection
  with `bar' it is generally traced to the WWII-era Army slang acronym
  FUBAR (`F***ed Up Beyond All Repair'), later bowdlerized to
  foobar. 
  It has been plausibly suggested that FUBAR  was influenced by German
  `furchtbar' (terrible). 
  It has also been reported out that 1960s computer manuals, in a
  usage influenced by Fortran's implicit-declaration feature,
  frequently used F00 (F followed by two zeros) in examples. 
\end{quote}

\section{Foo in Computer Science}
\label{sec:cosc}
Computer Science\footnote{and computing in general} has picked up on
the use of `foo' to stand for something, any element but no specific
element.
It sits outside the syntax of any language (including English) and is
able to descibe the syntax, hence the meta-syntatic nature of the
word.

\bibliographystyle{unsrt}
\bibliography{references}
\end{document}